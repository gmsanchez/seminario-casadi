% Copyright 2004 by Till Tantau <tantau@users.sourceforge.net>.
%
% In principle, this file can be redistributed and/or modified under
% the terms of the GNU Public License, version 2.
%
% However, this file is supposed to be a template to be modified
% for your own needs. For this reason, if you use this file as a
% template and not specifically distribute it as part of a another
% package/program, I grant the extra permission to freely copy and
% modify this file as you see fit and even to delete this copyright
% notice. 

\documentclass{beamer}

\usepackage[spanish]{babel}
\selectlanguage{spanish}
\usepackage[utf8]{inputenc}

% There are many different themes available for Beamer. A comprehensive
% list with examples is given here:
% http://deic.uab.es/~iblanes/beamer_gallery/index_by_theme.html
% You can uncomment the themes below if you would like to use a different
% one:
%\usetheme{AnnArbor}
%\usetheme{Antibes}
%\usetheme{Bergen}
%\usetheme{Berkeley}
%\usetheme{Berlin}
%\usetheme{Boadilla}
%\usetheme{boxes}
%\usetheme{CambridgeUS}
\usetheme{Copenhagen}
%\usetheme{Darmstadt}
%\usetheme{default}
%\usetheme{Frankfurt}
%\usetheme{Goettingen}
%\usetheme{Hannover}
%\usetheme{Ilmenau}
%\usetheme{JuanLesPins}
%\usetheme{Luebeck}
%\usetheme{Madrid}
%\usetheme{Malmoe}
%\usetheme{Marburg}
%\usetheme{Montpellier}
%\usetheme{PaloAlto}
%\usetheme{Pittsburgh}
%\usetheme{Rochester}
%\usetheme{Singapore}
%\usetheme{Szeged}
%\usetheme{Warsaw}

\title{Optimización con Python}

% A subtitle is optional and this may be deleted
\subtitle{Una introducción a CasADi}

\author{Guido ~Sanchez}
% - Give the names in the same order as the appear in the paper.
% - Use the \inst{?} command only if the authors have different
%   affiliation.

\institute[sinc(i)] % (optional, but mostly needed)
{
  Centro de Investigación en Señales, Sistemas e Inteligencia Computacional \\
  Universidad Nacional del Litoral
}
% - Use the \inst command only if there are several affiliations.
% - Keep it simple, no one is interested in your street address.

\date{Seminarios del sinc(i), 2015}
% - Either use conference name or its abbreviation.
% - Not really informative to the audience, more for people (including
%   yourself) who are reading the slides online

\subject{Optimización con Python}
% This is only inserted into the PDF information catalog. Can be left
% out. 

% If you have a file called "university-logo-filename.xxx", where xxx
% is a graphic format that can be processed by latex or pdflatex,
% resp., then you can add a logo as follows:

% \pgfdeclareimage[height=0.5cm]{university-logo}{university-logo-filename}
% \logo{\pgfuseimage{university-logo}}

% Delete this, if you do not want the table of contents to pop up at
% the beginning of each subsection:
%\AtBeginSubsection[]
%{
%  \begin{frame}<beamer>{Outline}
%    \tableofcontents[currentsection,currentsubsection]
%  \end{frame}
%}

% Let's get started
\begin{document}

\begin{frame}
  \titlepage
\end{frame}

\begin{frame}{Outline}
  \tableofcontents
  % You might wish to add the option [pausesections]
\end{frame}


\section{Introducción}

\subsection{¿Qué es CasADi?}

\begin{frame}{¿Qué es CasADi?}{}
  \begin{itemize}
  \item Es un framework open-source (LGPL) utilizado para la diferenciación algorítmica, la optimización numérica y control óptimo.
  \item Utiliza la sintaxis de los sistemas de álgebra (CAS) y permite construir expresiones simbólicas que pueden ser diferenciadas automáticamente de forma eficiente.
  \item Actualmente, es una herramienta general para optimización numérica basada en métodos de gradiente, con un fuerte enfoque hacia el control óptimo.
  \item Desarrollado por Joel Andersson y Joris Gillis en el Optimization in Engineering Center (OPTEC) de KU Leuven bajo supervisión de Moritz Diehl.
  \end{itemize}
\end{frame}

\subsection{¿Qué no es CasADi?}

\begin{frame}{¿Qué \textit{no} es CasADi?}
  \begin{itemize}
  \item No es un sistema de diferenciación algorítmica convencional que permita calcular derivadas a partir de código existente sin demasiadas modificaciones. Se debe reimplementar utilizando la sintaxis de CasADi.
  \item No es un sistema de álgebra. A pesar de que CasADi manipula expresiones simbólicas, sus capacidades son limitadas a comparación de una herramienta CAS.
  \item No es un solver de problemas de control óptimo, sino que trata de proveer con los bloques que permitan al usuario construir su solver.
  \end{itemize}
\end{frame}

\subsection{Componentes principales}

\begin{frame}{Componentes principales}
    \begin{itemize}
    \item Un CAS minimalista (como el Symbolic Toolbox de Matlab).
    \item Los algoritmos de diferenciación soportan:
        \begin{itemize}
            \item Modo hacia adelante (forward) y hacia atras (adjoint).
            \item Simbólico y numérico.
            \item Dos maneras de representar expresiones.
        \end{itemize}
    \item Interfaces a Ipopt, Sundials, (KNITRO, OOQP, qpOASES, CPLEX,
LAPACK, CSparse, ACADO Toolkit ...
    \item Front-ends para C++, Python, Octave y Matlab.
    \item Importa modelos de JModelica.org
    \end{itemize}
\end{frame}

\section{Manos a la obra}

\subsection{El framework simbólico}

\begin{frame}{Tipos de datos fundamentales}
    \begin{itemize}
    \item SX – tipo simbólico escalar.
    \item SXMatrix y DMatrix – matrices sparse.
    \item FX y clases derivadas – funciones de CasADi.
    \item MX – tipo simbólico matricial.
    \end{itemize}
\end{frame}


\subsection{Integración de ODE/DAE}

\begin{frame}{Integración de ODE/DAE}
    \begin{itemize}
    \item Suite de integradores open-source Sundials (ODE: CVodes / DAE: IDAS).
    \item Utilización: \texttt{integrator = casadi.Integrator(ode function)}.
    \item Casadi se encarga de armar las ecuaciones necesarias.
    \end{itemize}
\end{frame}

\subsection{Optimización}

\begin{frame}{Optimización}
\begin{equation}
\begin{array}{cl}
\underset{x \in \mathbb{R}^N}{\text{minimizar}}  & \displaystyle f(x) \\\\
\text{sujeto a}
& x_{min} \leq x \leq x_{max}   \\\\
& g_{min} \leq g(x) \leq g_{max}
\end{array}
\end{equation}

\begin{itemize}
\item Restricciones de igualdad ($x_{min} = x_{max}$) para algunos $x$.
\item Se formula utilizando solvers NLP (por ejemplo, IPOPT).
\end{itemize}

\end{frame}

\begin{frame}{Repaso I}
Sólo restricciones de igualdad

\begin{equation}
\begin{array}{cl}
\underset{x \in \mathbb{R}^N}{\text{minimizar}}  & \displaystyle f(x) \\
\text{sujeto a}
& g(x) = 0
\end{array}
\end{equation}

Para una solución óptima $x^*$ existen multiplicadores $\lambda^*$ tal que:

\begin{equation}
\begin{array}{rl}
 \nabla_x \mathcal{L}(x^*, \lambda^*) =& 0  \\
 g(x^*) =& 0 
\end{array}
\end{equation}

$\nabla_x \mathcal{L}(x,\lambda) = f(x) - \lambda^T g(x)$ es el Lagrangiano 

\end{frame}

\begin{frame}{Repaso II}

Con restricciones de igualdad y desigualdad

\begin{equation}
\begin{array}{cl}
\underset{x \in \mathbb{R}^N}{\text{minimizar}}  & \displaystyle f(x) \\
\text{sujeto a}
& g(x) = 0, h(x) \geq 0
\end{array}
\end{equation}

Condiciones de Karush-Kuhn-Tucker (KKT): Para una solución óptima $x^*$ existen multiplicadores $\lambda^*$, $\mu^*$ tal que:

\begin{equation}
\begin{array}{rl}
 \nabla_x \mathcal{L}(x^*, \lambda^*, \mu^*) =& 0  \\
 g(x^*) =& 0 \\
 h(x^*) \geq & 0 \\
 \mu^* \geq & 0 \\
 h(x^*)^T \mu^* =& 0
\end{array}
\end{equation}

$\nabla_x \mathcal{L}(x,\lambda,\mu) = f(x) - \lambda^T g(x) - \mu^T h(x)$ es el Lagrangiano 

\end{frame}

\begin{frame}{IPOPT}
    \begin{itemize}
    \item Optimizador open-source de punto interior: IPOPT\footnote{www.coin-or.org/Ipopt}.
    \item Resuelve NLP grandes (millones de variables/restricciones).
    \item Para resolver el sistema lineal se usa MA27, MA57, Mumps, Paradiso, ...
    \item Calcula $\frac{\partial g}{\partial x}$, $\nabla_x \mathcal{L}$ y $\nabla_x^2 \mathcal{L}$ utilizando algoritmos eficientes.
    \item Penaliza las desigualdades utilizando una funcion de barrera $\tau log(h(x))$,
        \begin{equation}
        \begin{array}{cl}
        \underset{x \in \mathbb{R}^N}{\text{minimizar}}  & \displaystyle f(x) - \tau log(h(x)) \\
        \text{sujeto a} & g(x) = 0
        \end{array}
        \end{equation}
    \end{itemize}
\end{frame}

\subsection{Ejemplo de NLP}

\begin{frame}{Notebook de iPython}
\begin{itemize}
\item ¡Manos a la obra!
\end{itemize}
\end{frame}

\section*{Conclusiones}

\begin{frame}{Conclusiones}
  \begin{itemize}
  \item Permite formular problemas de optimización de forma simple.
  \item El framework simbólico permite plantear el algoritmo con formulaciones ``de libro''.
  \item Los solver utilizados necesitan muy poco tiempo para resolver problemas de optimización no lineales.
  \end{itemize}
\end{frame}

\section*{Preguntas}

\begin{frame}{}
    \begin{center}
    ¿Preguntas?
    \end{center}
\end{frame}

% All of the following is optional and typically not needed. 
\appendix
\section<presentation>*{\appendixname}
\subsection<presentation>*{Más información}

\begin{frame}{Más información}

    \begin{itemize}
    \item http://www.casadi.org/
    \item https://github.com/casadi/casadi
    \item http://casadi.sourceforge.net/users\_guide/html/casadi-users\_guide.html
    \item Alternativa: CVXOPT - http://cvxopt.org/
    \end{itemize}

\end{frame}

\end{document}


